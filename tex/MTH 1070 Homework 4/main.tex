\documentclass[16pt]{article}
\usepackage[margin=0.5in]{geometry} 
\usepackage{amsmath,amsthm,amssymb,graphicx,mathtools,tikz,hyperref, enumitem}
\usetikzlibrary{positioning}
\newcommand{\n}{\mathbb{N}}
\newcommand{\z}{\mathbb{Z}}
\newcommand{\q}{\mathbb{Q}}
\newcommand{\cx}{\mathbb{C}}
\newcommand{\real}{\mathbb{R}}
\newcommand{\field}{\mathbb{F}}
\newcommand{\ita}[1]{\textit{#1}}
\newcommand{\com}[2]{#1\backslash#2}
\newcommand{\oneton}{\{1,2,3,...,n\}}
\newcommand\idea[1]{\begin{gather*}#1\end{gather*}}
\newcommand\ef{\ita{f} }
\newcommand\eff{\ita{f}}
\newcommand\proofs[1]{\begin{proof}#1\end{proof}}
\newcommand\inv[1]{#1^{-1}}
\newcommand\setb[1]{\{#1\}}
\newcommand\en{\ita{n }}
\newcommand{\vbrack}[1]{\langle #1\rangle}
\newenvironment{theorem}[2][Theorem]{\begin{trivlist}
\item[\hskip \labelsep {\bfseries #1}\hskip \labelsep {\bfseries #2.}]}{\end{trivlist}}
\newenvironment{lemma}[2][Lemma]{\begin{trivlist}
\item[\hskip \labelsep {\bfseries #1}\hskip \labelsep {\bfseries #2.}]}{\end{trivlist}}
\newenvironment{exercise}[2][Exercise]{\begin{trivlist}
\item[\hskip \labelsep {\bfseries #1}\hskip \labelsep {\bfseries #2.}]}{\end{trivlist}}
\newenvironment{reflection}[2][Reflection]{\begin{trivlist}
\item[\hskip \labelsep {\bfseries #1}\hskip \labelsep {\bfseries #2.}]}{\end{trivlist}}
\newenvironment{proposition}[2][Proposition]{\begin{trivlist}
\item[\hskip \labelsep {\bfseries #1}\hskip \labelsep {\bfseries #2.}]}{\end{trivlist}}
\newenvironment{corollary}[2][Corollary]{\begin{trivlist}
\item[\hskip \labelsep {\bfseries #1}\hskip \labelsep {\bfseries #2.}]}{\end{trivlist}}
\hypersetup{colorlinks, linkcolor=blue}

\begin{document}
\Large
\date{}
\title{\LARGE MTH 1070 Homework 4}
\author{\LARGE Ryan Heffelman\\\LARGE Aaron Trautwein \\\LARGE 
MTH 1070 \\\LARGE Functions, Analysis, and Graphs. \\\LARGE January \(18^{th}\) 2024} 
\maketitle

\section{Simplify the given expressions such that only positive exponents are used.}
\LARGE
    \begin{enumerate} \begin{enumerate} \begin{enumerate}
    
    
        \item \(\frac{5^2 x^{-2} y^3}{5^4 x^5 y^2}\) = \(5^{2-4}\) \(x^{-2-5}\) \(y^{3-2}\) = \(5^{-2}\) \(x^{-7}\) \(y^1\) = \(\frac{y}{5^2 x^7}\) = \(\frac{y}{25 x^7}\) .

        \item \(\frac{3^{\frac{2}{3}} a^4 b^{\frac{4}{5}}}{3^{\frac{1}{3}} a^{\frac{2}{3}} b^{\frac{1}{5}}}\) =  \(3^{\frac{1}{3}}\) \(a^{4 - \frac{2}{3}}\) \(b^{\frac{4}{5} - \frac{1}{4}}\) = \(\sqrt[3]{3}\) \(a^{10/3}\) \(b^{3/5}\) or \(3^{\frac{1}{3}}\) \(a^{10/3}\) \(b^{3/5}\) . 
        
    \end{enumerate} \end{enumerate} \end{enumerate}
\section{If \(\log _{4} 6 = a\) and \(\log _{4} 5 = b\), find the following in terms of $a$ and $b$.}
    \begin{enumerate} \begin{enumerate} \begin{enumerate}
        \item \(\log _{4} 30\) = \(\log _{4}(6\cdot 5)\) = \(\log _{4} 6\) + \(\log _{4} 5\) = \(a\) + \(b\).
        \item \(\log _{4} 150\). If a + b = \(\log _{4} 30\) and \(\frac{150}{30}\) = 5 then \(5(a + b)\) = \(\log _{4} 150\).
        \item Similarly, \(\log _{4} 80\) = \(2(a + b)\) + \(\frac{2}{3}(a + b)\). Because \(2(a + b)\) = 60 and \(\frac{2}{3}(a + b)\) = 20, 60 + 20 = 80.
    \end{enumerate} \end{enumerate} \end{enumerate}
\section{Graph and find the Domain and Range of the \(\log _{a} x\) through \(f^{-1}(x) = a^x\).}
    \begin{enumerate} \begin{enumerate} \begin{enumerate}
    \item \(\log _5 x = f(x).\) \(f^{-1}(x) = 5^x\)
    \begin{itemize}
        \item Domain of \(f(x)\) = (0, \(\infty\)).
        \item Range \(f(x)\) = \(\mathbb{R}\).
    \end{itemize}
    \end{enumerate} \end{enumerate} \end{enumerate}
\end{document}