\documentclass[16pt]{article}
\usepackage[margin=0.7in]{geometry} 
\usepackage{amsmath,amsthm,amssymb,graphicx,mathtools,tikz,hyperref, enumitem}
\usetikzlibrary{positioning}
\newcommand{\n}{\mathbb{N}}
\newcommand{\z}{\mathbb{Z}}
\newcommand{\q}{\mathbb{Q}}
\newcommand{\cx}{\mathbb{C}}
\newcommand{\real}{\mathbb{R}}
\newcommand{\field}{\mathbb{F}}
\newcommand{\ita}[1]{\textit{#1}}
\newcommand{\com}[2]{#1\backslash#2}
\newcommand{\oneton}{\{1,2,3,...,n\}}
\newcommand\idea[1]{\begin{gather*}#1\end{gather*}}
\newcommand\ef{\ita{f} }
\newcommand\eff{\ita{f}}
\newcommand\proofs[1]{\begin{proof}#1\end{proof}}
\newcommand\inv[1]{#1^{-1}}
\newcommand\setb[1]{\{#1\}}
\newcommand\en{\ita{n }}
\newcommand{\vbrack}[1]{\langle #1\rangle}
\newenvironment{theorem}[2][Theorem]{\begin{trivlist}
\item[\hskip \labelsep {\bfseries #1}\hskip \labelsep {\bfseries #2.}]}{\end{trivlist}}
\newenvironment{lemma}[2][Lemma]{\begin{trivlist}
\item[\hskip \labelsep {\bfseries #1}\hskip \labelsep {\bfseries #2.}]}{\end{trivlist}}
\newenvironment{exercise}[2][Exercise]{\begin{trivlist}
\item[\hskip \labelsep {\bfseries #1}\hskip \labelsep {\bfseries #2.}]}{\end{trivlist}}
\newenvironment{reflection}[2][Reflection]{\begin{trivlist}
\item[\hskip \labelsep {\bfseries #1}\hskip \labelsep {\bfseries #2.}]}{\end{trivlist}}
\newenvironment{proposition}[2][Proposition]{\begin{trivlist}
\item[\hskip \labelsep {\bfseries #1}\hskip \labelsep {\bfseries #2.}]}{\end{trivlist}}
\newenvironment{corollary}[2][Corollary]{\begin{trivlist}
\item[\hskip \labelsep {\bfseries #1}\hskip \labelsep {\bfseries #2.}]}{\end{trivlist}}
\hypersetup{colorlinks, linkcolor=blue}

\newcounter{itemnumber}

\begin{document}
\large
\date{}
\title{\Large Assignment 2 Part 2 \\ Ryan Heffelman \\ February 13^\(th\), 2024 \\ MTH 1240 \\ Sara Jensen}
\maketitle
\begin{enumerate}
    \item[\textbf{1.1}]
    \item[]
    \begin{enumerate}
        \item[\textbf{\#4.}] \{1, 2, 3, 4, 5, 6, 7\}
        \item[\textbf{\#7.}] \{-3, -2\}
        \item[\textbf{\#12.}] \{-2, -1, 0, 1, 2\}
        \item[\textbf{\#29.}] 3
        \item[\textbf{\#30.}] 5
        \item[\textbf{\#36.}] 3
        \item[\textbf{\#37.}] 0
    \end{enumerate}
    \item[\textbf{2.7}]
    \item[]
    \begin{enumerate}
        \item[\textbf{\#9.}] For all $n$ in the set of integers, there exists $m$ in the set of integers such that $m = n + 5$. This is a true sentence.
        \item[\textbf{\#10.}] There exists $m$ in the set of integers, such that for all $n$ in the set of integers, $m = n + 5$. \\
        If you view ',' as the $\land$ connective, then this sentence is equivalent to the previous one per the commutativity property of logical conjunction. So the truthfulness of \#10 is implied to be truthful by \#9.
    \end{enumerate}
\end{enumerate}

\end{document}