\documentclass[16pt]{article}
\usepackage[margin=0.5in]{geometry} 
\usepackage{amsmath,amsthm,amssymb,graphicx,mathtools,tikz,hyperref, enumitem}
\usetikzlibrary{positioning}
\newcommand{\n}{\mathbb{N}}
\newcommand{\z}{\mathbb{Z}}
\newcommand{\q}{\mathbb{Q}}
\newcommand{\cx}{\mathbb{C}}
\newcommand{\real}{\mathbb{R}}
\newcommand{\field}{\mathbb{F}}
\newcommand{\ita}[1]{\textit{#1}}
\newcommand{\com}[2]{#1\backslash#2}
\newcommand{\oneton}{\{1,2,3,...,n\}}
\newcommand\idea[1]{\begin{gather*}#1\end{gather*}}
\newcommand\ef{\ita{f} }
\newcommand\eff{\ita{f}}
\newcommand\proofs[1]{\begin{proof}#1\end{proof}}
\newcommand\inv[1]{#1^{-1}}
\newcommand\setb[1]{\{#1\}}
\newcommand\en{\ita{n }}
\newcommand{\vbrack}[1]{\langle #1\rangle}
\newenvironment{theorem}[2][Theorem]{\begin{trivlist}
\item[\hskip \labelsep {\bfseries #1}\hskip \labelsep {\bfseries #2.}]}{\end{trivlist}}
\newenvironment{lemma}[2][Lemma]{\begin{trivlist}
\item[\hskip \labelsep {\bfseries #1}\hskip \labelsep {\bfseries #2.}]}{\end{trivlist}}
\newenvironment{exercise}[2][Exercise]{\begin{trivlist}
\item[\hskip \labelsep {\bfseries #1}\hskip \labelsep {\bfseries #2.}]}{\end{trivlist}}
\newenvironment{reflection}[2][Reflection]{\begin{trivlist}
\item[\hskip \labelsep {\bfseries #1}\hskip \labelsep {\bfseries #2.}]}{\end{trivlist}}
\newenvironment{proposition}[2][Proposition]{\begin{trivlist}
\item[\hskip \labelsep {\bfseries #1}\hskip \labelsep {\bfseries #2.}]}{\end{trivlist}}
\newenvironment{corollary}[2][Corollary]{\begin{trivlist}
\item[\hskip \labelsep {\bfseries #1}\hskip \labelsep {\bfseries #2.}]}{\end{trivlist}}
\hypersetup{colorlinks, linkcolor=blue}

\begin{document}
\Large
\date{}
\title{\LARGE MTH 1070 presentation notes}
\author{\LARGE Ryan Heffelman\\\LARGE Aaron Trautwein \\\LARGE 
MTH 1070 \\\LARGE Functions, Analysis, and Graphs. \\\LARGE January \(8^{th}\) 2024} 
\maketitle

\section{Ada Lovelace}
\subsection{-Location}
London, United Kingdom.

\subsection{-Time Period}
December 1815 to November 1852, died aged 36

\subsection{-Education}
She married rich which is what allowed her to pursue math as a woman in that time period, and her mother instilled in her a strong passion for mathematics and logic. She was privately educated by several notable mathematicians. One of which being Augustus De Morgan, as in the De Morgan Laws. Another being Mary Somerville, a very prominent female mathematician from the same time/place, who is deserving of a presentation herself.

\subsection{-What did she do?}
Her connection with Mary Somerville is what introduced her to a mathematics professor from Cambridge named Charles Babbage in 1833. Babbage had a peculiar idea of how to make a certain type of machine for computation which he called the ``Analytical Engine''. It was essentially the earliest theoretical Digital Computer, but it was never able to be built due to a lack of funds and understanding of what it even did. Modern Computer Scientists would call it a ``Discrete State Machine'', but it was unlike anything ever thought of before, it was way ahead of its time. The first physical digital computer was made in 1945, almost 100 years after Ada's death. Ada was one of the few people who saw and understood the potential of Babbage's machine, and she ended up working closely with babbage and contributed heavily to the theoretical work of it. One of the things she did was she created a ``program'' that computed Bernoulli numbers that might have worked on the babbage analytical engine if it were ever built. This was the first computer program ever written, making her the first ever computer programmer. In addition she translated and expanded on an article from a renowned Italian Mathematician's article on the analytical engine, which sounds less important than it is, in reality this work essentially made her the worlds foremost expert on the analytical engine, besides charles babbage himself. She was also featured prominently by Alan Turing in his famous Computing Machinery and Intelligence paper which inspired The Imitation Game movie, coined the term "turing machine", and "discrete state machine". She died at age 36 in 1852 from cancer.
\end{document}