\documentclass[16pt]{article}
\usepackage[margin=0.7in]{geometry} 
\usepackage{amsmath,amsthm,amssymb,graphicx,mathtools,tikz,hyperref, enumitem}
\usetikzlibrary{positioning}
\newcommand{\n}{\mathbb{N}}
\newcommand{\z}{\mathbb{Z}}
\newcommand{\q}{\mathbb{Q}}
\newcommand{\cx}{\mathbb{C}}
\newcommand{\real}{\mathbb{R}}
\newcommand{\field}{\mathbb{F}}
\newcommand{\ita}[1]{\textit{#1}}
\newcommand{\com}[2]{#1\backslash#2}
\newcommand{\oneton}{\{1,2,3,...,n\}}
\newcommand\idea[1]{\begin{gather*}#1\end{gather*}}
\newcommand\ef{\ita{f} }
\newcommand\eff{\ita{f}}
\newcommand\proofs[1]{\begin{proof}#1\end{proof}}
\newcommand\inv[1]{#1^{-1}}
\newcommand\setb[1]{\{#1\}}
\newcommand\en{\ita{n }}
\newcommand{\vbrack}[1]{\langle #1\rangle}
\newenvironment{theorem}[2][Theorem]{\begin{trivlist}
\item[\hskip \labelsep {\bfseries #1}\hskip \labelsep {\bfseries #2.}]}{\end{trivlist}}
\newenvironment{lemma}[2][Lemma]{\begin{trivlist}
\item[\hskip \labelsep {\bfseries #1}\hskip \labelsep {\bfseries #2.}]}{\end{trivlist}}
\newenvironment{exercise}[2][Exercise]{\begin{trivlist}
\item[\hskip \labelsep {\bfseries #1}\hskip \labelsep {\bfseries #2.}]}{\end{trivlist}}
\newenvironment{reflection}[2][Reflection]{\begin{trivlist}
\item[\hskip \labelsep {\bfseries #1}\hskip \labelsep {\bfseries #2.}]}{\end{trivlist}}
\newenvironment{proposition}[2][Proposition]{\begin{trivlist}
\item[\hskip \labelsep {\bfseries #1}\hskip \labelsep {\bfseries #2.}]}{\end{trivlist}}
\newenvironment{corollary}[2][Corollary]{\begin{trivlist}
\item[\hskip \labelsep {\bfseries #1}\hskip \labelsep {\bfseries #2.}]}{\end{trivlist}}
\hypersetup{colorlinks, linkcolor=blue}

\newcounter{itemnumber}

\begin{document}
\large
\date{}
\author{}
\title{\Large Assignment 1 \\ Ryan Heffelman \\ February 3rd, 2024 \\ MTH 1240 \\ Sara Jensen}
\maketitle

\begin{enumerate}
    \item[$\textbf{2.1  }$]  
    \begin{enumerate}
        \item[\textbf{\#1.}] Every real number is an even integer.
        \begin{itemize}
            \item A false statement.
        \end{itemize}
        \item[\textbf{\#9.}] $\cos x = -1$.
        \begin{itemize}
            \item Not a statement.
            \item For example: true when \(x = \pi\), false when \(x = 2\pi\).
        \end{itemize}
        \item[\textbf{\#13.}] Either $x$ is a multiple of 7, or it is not.
        \begin{itemize}
            \item A tautology.
            \item \(P\) \(\oplus\)  \(!P = T\) 
        \end{itemize}
    \end{enumerate}
    \item[\textbf{2.2  }]
    \begin{enumerate}
        \item[\textbf{\#1.}] The number 8 is both even and a power of 2.
        \begin{enumerate}
            \item[1.] If $P$ is ``8 is even'',
            \item[2.] and $Q$ is ``8 is a power of 2'',
            \item[3.] then \(P \land Q\).
        \end{enumerate}
        \item[\textbf{\#8.}] At least one of the numbers $x$ and $y$ equals 0.
        \begin{enumerate}
            \item[1.] If $P$ is ``$x$ = 0'',
            \item[2.] and $Q$ is ``$y$ = 0'',
            \item[3.] then \(P \lor Q\).
        \end{enumerate}
    \end{enumerate}
    \\
    \item[\textbf{2.3  }]
    \begin{tabular}{ |c|c|c|c|c|c|} 
        \hline
        $P$ & $Q$ & $R$ & $\sim (P\lor Q)\lor (\sim P)$ & $P \lor (Q \land \sim R)$ & $\sim (\sim P \lor \sim Q)$\\
        \hline
        T & T & T & F & T & T\\ 
        \hline
        T & T & F & F & T & T\\ 
        \hline
        T & F & T & F & T & F\\ 
        \hline
        T & F & F & F & T & F\\
        \hline
        F & T & T & T & F & F\\
        \hline
        F & T & F & T & T & F\\
        \hline
        F & F & T & T & F & F\\
        \hline
        F & F & F & T & F & F\\
        \hline
    \end{tabular}
    
\end{enumerate}

\end{document}