\documentclass[16pt]{article}
\usepackage[margin=0.7in]{geometry} 
\usepackage{amsmath,amsthm,amssymb,graphicx,mathtools,tikz,hyperref, enumitem}
\usetikzlibrary{positioning}
\newcommand{\n}{\mathbb{N}}
\newcommand{\z}{\mathbb{Z}}
\newcommand{\q}{\mathbb{Q}}
\newcommand{\cx}{\mathbb{C}}
\newcommand{\real}{\mathbb{R}}
\newcommand{\field}{\mathbb{F}}
\newcommand{\ita}[1]{\textit{#1}}
\newcommand{\com}[2]{#1\backslash#2}
\newcommand{\oneton}{\{1,2,3,...,n\}}
\newcommand\idea[1]{\begin{gather*}#1\end{gather*}}
\newcommand\ef{\ita{f} }
\newcommand\eff{\ita{f}}
\newcommand\proofs[1]{\begin{proof}#1\end{proof}}
\newcommand\inv[1]{#1^{-1}}
\newcommand\setb[1]{\{#1\}}
\newcommand\en{\ita{n }}
\newcommand{\vbrack}[1]{\langle #1\rangle}
\newenvironment{theorem}[2][Theorem]{\begin{trivlist}
\item[\hskip \labelsep {\bfseries #1}\hskip \labelsep {\bfseries #2.}]}{\end{trivlist}}
\newenvironment{lemma}[2][Lemma]{\begin{trivlist}
\item[\hskip \labelsep {\bfseries #1}\hskip \labelsep {\bfseries #2.}]}{\end{trivlist}}
\newenvironment{exercise}[2][Exercise]{\begin{trivlist}
\item[\hskip \labelsep {\bfseries #1}\hskip \labelsep {\bfseries #2.}]}{\end{trivlist}}
\newenvironment{reflection}[2][Reflection]{\begin{trivlist}
\item[\hskip \labelsep {\bfseries #1}\hskip \labelsep {\bfseries #2.}]}{\end{trivlist}}
\newenvironment{proposition}[2][Proposition]{\begin{trivlist}
\item[\hskip \labelsep {\bfseries #1}\hskip \labelsep {\bfseries #2.}]}{\end{trivlist}}
\newenvironment{corollary}[2][Corollary]{\begin{trivlist}
\item[\hskip \labelsep {\bfseries #1}\hskip \labelsep {\bfseries #2.}]}{\end{trivlist}}
\hypersetup{colorlinks, linkcolor=blue}

\newcounter{itemnumber}

\begin{document}
\large
\date{}
\title{\Large Assignment 1 \\ Ryan Heffelman \\ February 6^\(th\), 2024 \\ MTH 1240 \\ Sara Jensen}
\maketitle
\begin{enumerate}
    \item[\textbf{2.3  }]  Without changing their meanings, convert each of the following sentences into a sentence having the form ``If $P$, then $Q$''.
    \begin{enumerate}
        \item[\textbf{\#2.}] For a function to be continuous, it is sufficient that it is differentiable.
        \begin{enumerate}
            \item[1.] If $P$ is ``a function is differentiable''
            \item[2.] and $Q$ is ``a function is continuous''
            \item[3.] then $P\Rightarrow Q$.
        \end{enumerate}
        \item[\textbf{\#4.}] A function is rational if it is a polynomial.
        \begin{enumerate}
            \item[1.] If $P$ is ``a function is a polynomial'',
            \item[2.] and $Q$ is ``it is rational'',
            \item[3.] then $P\Rightarrow Q$.
        \end{enumerate}
        \item[\textbf{\#5.}] An integer is divisible by 8 only if it is divisible by 4.
        \begin{enumerate}
            \item[1. ] If $P$ is ``A number is divisible by 8''
            \item[2. ] and $Q$ is ``the number is divisible by 4''
            \item[3. ] then $P \Rightarrow Q$.
            \item[Note: ] This happens to work in this case, but rearranging the logic like this doesn't work in all cases.
        \end{enumerate}
        \\
    \end{enumerate}
    \item[\textbf{2.5  }]
    \begin{enumerate}
        \item[\textbf{\#2.}]
        \begin{tabular}{|c|c|c|}
            \hline
            $Q$ & $R$ & $(Q\lor R)\Leftrightarrow (R\land Q)$\\
            \hline
            T & T & T \\
            \hline
            T & F & F \\
            \hline
            F & T & F \\
            \hline
            F & F & T \\
            \hline
        \end{tabular}
        \\ \\
        \item[\textbf{\#10.}] $P$ and $Q$ are True, while $R$ and $S$ are False. In order for the statement to be false the consequent must be false while the antecedent must be true, for the consequent to be false $R$ and $S$ must be false. Once you know that, for the antecedent to be true $P$ and $Q$ must be true.
    \end{enumerate}

    \item[\textbf{2.6  }]
    \begin{enumerate}
        \item[]
        \item[\textbf{\#2.}]
        \begin{tabular}{|c|c|c|c|c|}
            \hline
            $P$ & $Q$ & $R$ & $P\lor (Q\land R)$ & $(P\lor Q)\land (P\lor R)$\\
            \hline
            T & T & T & T & T\\ 
            \hline
            T & T & F & T & T\\ 
            \hline
            T & F & T & T & T\\ 
            \hline
            T & F & F & T & T\\
            \hline
            F & T & T & T & T\\
            \hline
            F & T & F & F & F\\
            \hline
            F & F & T & F & F\\
            \hline
            F & F & F & F & F\\
            \hline
        \end{tabular}
        \\
        \item[\textbf{\#12.}] The statements are logically equivalent. \\
        \begin{tabular}{|c|c|c|c|}
            \hline
            $P$ & $Q$ & $\sim (P\Rightarrow Q)$ & $P\land \sim Q$\\
            \hline
            T & T & F & F \\
            \hline
            T & F & T & T \\
            \hline
            F & T & F & F \\ 
            \hline
            F & F & F & F \\
            \hline
        \end{tabular}
    \end{enumerate}
    \\ \\
    \item[\textbf{Witno:  }] \\
    \begin{enumerate}
    \item[]
        \item[\textbf{2.6 (c)}]
        \begin{tabular}{|c|c|c|c|}
        \hline
        $p$ & $q$ & $r$ & $((p \bigoplus q) \bigoplus r)$\\
        \hline
        T & T & T & T \\
        \hline
        T & T & F & F \\
        \hline
        T & F & T & F \\
        \hline
        T & F & F & T \\
        \hline
        F & T & T & F \\
        \hline
        F & T & F & T \\
        \hline
        F & F & T & T \\
        \hline
        F & F & F & F \\
        \hline
    \end{tabular}
    \end{enumerate}
    \item[\textbf{2.14}]
    \item[] \textbf{c}, $(\sim p \Leftrightarrow \sim q) \neq (p \bigoplus q)$
    
    
\end{enumerate}

\end{document}