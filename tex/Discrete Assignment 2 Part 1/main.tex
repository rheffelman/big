\documentclass[16pt]{article}
\usepackage[margin=0.7in]{geometry} 
\usepackage{amsmath,amsthm,amssymb,graphicx,mathtools,tikz,hyperref, enumitem}
\usetikzlibrary{positioning}
\newcommand{\n}{\mathbb{N}}
\newcommand{\z}{\mathbb{Z}}
\newcommand{\q}{\mathbb{Q}}
\newcommand{\cx}{\mathbb{C}}
\newcommand{\real}{\mathbb{R}}
\newcommand{\field}{\mathbb{F}}
\newcommand{\ita}[1]{\textit{#1}}
\newcommand{\com}[2]{#1\backslash#2}
\newcommand{\oneton}{\{1,2,3,...,n\}}
\newcommand\idea[1]{\begin{gather*}#1\end{gather*}}
\newcommand\ef{\ita{f} }
\newcommand\eff{\ita{f}}
\newcommand\proofs[1]{\begin{proof}#1\end{proof}}
\newcommand\inv[1]{#1^{-1}}
\newcommand\setb[1]{\{#1\}}
\newcommand\en{\ita{n }}
\newcommand{\vbrack}[1]{\langle #1\rangle}
\newenvironment{theorem}[2][Theorem]{\begin{trivlist}
\item[\hskip \labelsep {\bfseries #1}\hskip \labelsep {\bfseries #2.}]}{\end{trivlist}}
\newenvironment{lemma}[2][Lemma]{\begin{trivlist}
\item[\hskip \labelsep {\bfseries #1}\hskip \labelsep {\bfseries #2.}]}{\end{trivlist}}
\newenvironment{exercise}[2][Exercise]{\begin{trivlist}
\item[\hskip \labelsep {\bfseries #1}\hskip \labelsep {\bfseries #2.}]}{\end{trivlist}}
\newenvironment{reflection}[2][Reflection]{\begin{trivlist}
\item[\hskip \labelsep {\bfseries #1}\hskip \labelsep {\bfseries #2.}]}{\end{trivlist}}
\newenvironment{proposition}[2][Proposition]{\begin{trivlist}
\item[\hskip \labelsep {\bfseries #1}\hskip \labelsep {\bfseries #2.}]}{\end{trivlist}}
\newenvironment{corollary}[2][Corollary]{\begin{trivlist}
\item[\hskip \labelsep {\bfseries #1}\hskip \labelsep {\bfseries #2.}]}{\end{trivlist}}
\hypersetup{colorlinks, linkcolor=blue}

\newcounter{itemnumber}

\begin{document}
\large
\date{}
\title{\Large Assignment 3 \\ Ryan Heffelman \\ February 10^\(th\), 2024 \\ MTH 1240 \\ Sara Jensen}
\maketitle
\begin{enumerate}
    \item[\textbf{2.9  }]  Translate each of the following sentences into symbolic logic.
    \begin{enumerate}
        \item[\textbf{\#7.}] There exists a real number $a$ for which $a + x = x$ for every real number x.
        \begin{enumerate}
            \item[A: ] $\forall x \in \textbb{R}, \exists a \in \textbb{R} \vert a + x = x$
        \end{enumerate}
        \item[\textbf{\#8.}] I don't eat anything that has a face.
        \begin{enumerate}
            \item[1.] If $P$ is the set of things that have a face,
            \item[2.] and $Q$ evaluates to true when I eat something and to false when I don't eat something,
            \item[3.] then $(\forall x \in P) \Rightarrow (Q \equiv F)$
        \end{enumerate}
        \item[\textbf{\#9.}] If $x$ is a rational number and $x \neq 0$ then $\tan(x)$ is not a rational number.
        \begin{enumerate}
            \item[A: ] $(x \in \mathbb{Q} \land x \neq 0) \Rightarrow (\tan(x) \notin \mathbb{Q})$
        \end{enumerate}
    \end{enumerate}
    \item[\textbf{2.10  }] Negate the following sentences.
    \begin{enumerate}
        \item[\#1.] The number x is not positive, and the number y is positive.
        \item[\#6.] There does not exist a real number $a$ for which $a + x = x$ for every real number $x$.
        \item[\#7.] I eat anything that has a face.
        \item[\#8.] ($x$ is rational $\land$ $x \neq 0$) $\land \tan(x)$ is a rational number. 
    \end{enumerate}
    \item[\textbf{Carroll}]
    \begin{enumerate}
        \item[1.] $\sim P \Rightarrow \sim V$
        \item[2.] $R \Rightarrow T$
        \item[3.] $P \Rightarrow S$
        \item[4.] $\sim R \Rightarrow \sim U$
        \item[5.] $T \Rightarrow V$
    \end{enumerate}
    \begin{enumerate}
        \item[a)] From 2 and 5 we get that all wine drinkers can be trusted.
        \item[b)] From 4 we get that all pawnbrokers are wine drinkers.
        \item[c)] From a and b we get that all pawnbrokers can be trusted.
        \item[d)] From 1 and c if $V$ is true then $P$ must be true, so all pawnbrokers keep their promises.
        \item[e)] From 3 and d all pawnbrokers are honest.
    \end{enumerate}
\end{enumerate}

\end{document}