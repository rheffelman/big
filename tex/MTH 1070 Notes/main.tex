\documentclass[12pt]{article}
\usepackage[margin=1in]{geometry} 
\usepackage{amsmath,amsthm,amssymb,graphicx,mathtools,tikz,hyperref, enumitem}
\usetikzlibrary{positioning}
\newcommand{\n}{\mathbb{N}}
\newcommand{\z}{\mathbb{Z}}
\newcommand{\q}{\mathbb{Q}}
\newcommand{\cx}{\mathbb{C}}
\newcommand{\real}{\mathbb{R}}
\newcommand{\field}{\mathbb{F}}
\newcommand{\ita}[1]{\textit{#1}}
\newcommand{\com}[2]{#1\backslash#2}
\newcommand{\oneton}{\{1,2,3,...,n\}}
\newcommand\idea[1]{\begin{gather*}#1\end{gather*}}
\newcommand\ef{\ita{f} }
\newcommand\eff{\ita{f}}
\newcommand\proofs[1]{\begin{proof}#1\end{proof}}
\newcommand\inv[1]{#1^{-1}}
\newcommand\setb[1]{\{#1\}}
\newcommand\en{\ita{n }}
\newcommand{\vbrack}[1]{\langle #1\rangle}
\newenvironment{theorem}[2][Theorem]{\begin{trivlist}
\item[\hskip \labelsep {\bfseries #1}\hskip \labelsep {\bfseries #2.}]}{\end{trivlist}}
\newenvironment{lemma}[2][Lemma]{\begin{trivlist}
\item[\hskip \labelsep {\bfseries #1}\hskip \labelsep {\bfseries #2.}]}{\end{trivlist}}
\newenvironment{exercise}[2][Exercise]{\begin{trivlist}
\item[\hskip \labelsep {\bfseries #1}\hskip \labelsep {\bfseries #2.}]}{\end{trivlist}}
\newenvironment{reflection}[2][Reflection]{\begin{trivlist}
\item[\hskip \labelsep {\bfseries #1}\hskip \labelsep {\bfseries #2.}]}{\end{trivlist}}
\newenvironment{proposition}[2][Proposition]{\begin{trivlist}
\item[\hskip \labelsep {\bfseries #1}\hskip \labelsep {\bfseries #2.}]}{\end{trivlist}}
\newenvironment{corollary}[2][Corollary]{\begin{trivlist}
\item[\hskip \labelsep {\bfseries #1}\hskip \labelsep {\bfseries #2.}]}{\end{trivlist}}
\hypersetup{colorlinks, linkcolor=blue}

\begin{document}
\date{January \(4^{th}\) 2024}
\title{Pre-Calc Notes}
\author{Ryan Heffelman\\Aaron Trautwein \\
MTH 1070 \\Functions, Analysis, and Graphs.} 
\maketitle

\section{Day 1, 1-4-2024}
\subsection{Contact Info}
\begin{enumerate}

    \item Office Phone: (262)-551-5873
    
    \item Home Phone: (262)-551-9915
    
    \item Email: atrautwein@carthage.edu
    
\end{enumerate}
\subsection{Notes}
\begin{enumerate}

    \item $\mathbb{N}$ = Set of natural/counting numbers = \{1, 2, 3, 4, 5, ...\}
    \item $\mathbb{Z}$ = Set of integers = \{..., -3, -2, -1, 0, 1, 2, 3, ...\}. Z comes from the German word "zahlen", which means "to count"
    \item $\mathbb{Q}$ = Set of rational numbers = \{\(\frac{p}{q}\): p, q\(\in\)$\mathbb{Z}$ and q \(\neq\) 0\}
    \item $\mathbb{I}$ = Set of irrational numbers = every real number that is not a rational number = \(\{\sqrt{2}, \frac{\sqrt{3} + 1}{4}, e, \pi, \sqrt{-5}\}\)
    \item $\mathbb{R}$ = Set of real numbers = every number along real number line
    \item $\mathbb{C}$ = Set of complex numbers.
    \item $\mathbb{U}$ = The universal Set, the set that contains all possible elements in the space you are working in.
    \item $\emptyset$ = The empty set, the set that contains no elements at all.
    \item The union of two sets A and B, denoted A\(\cup\)B, is the set that contains every element in A combined with every element in B.
    \item The intersection of two sets A and B, denoted A\(\cap\)B, is the set of elements that appears in both sets.
    \item The complement of A, denoted A', is the set of all elements in $\mathbb{U}$ that are not in A.
    \item A -B, or A not B, is the set of elements in A, not in B.
    
\end{enumerate}
\subsection{Assignments}
\begin{enumerate}

    \item Quiz on Monday 1-8-2024 on Sets. 
    
    \item Homework: Memorize, come up with examples of each of the operators. make a venn diagram. a union b, a intersect b, a complement, a -b, venn diagram.
    
\end{enumerate}
\section{Day 2, 1-5-2024}
\subsection{Definitions:}
\begin{enumerate}

    \item\textbf{Cartesian Product: } The cartesian product of sets A and B is the set of all ordered 2-tuples with a \(\in\) A and b \(\in\) B, that is denoted by A\(\times\)B. A\(\times\)B = \{(a, b): a \(\in\) A, b \(\in\) B\}
    \begin{enumerate}\begin{enumerate}
        \item A = \{1, 2, 3\}, B = \{1, 2\}. A\(\times\)B = \{(1, 1), (1, 2), (2, 1), (2, 2), (3, 1), (3, 2)\}
    \end{enumerate}\end{enumerate}
    \item\textbf{Order: }The order or cardinality of a set A, denoted \(\vert\)A\(\vert\) is the number of elements in A.
    \begin{enumerate}\begin{enumerate}
        \item A = \{1, 2, 3\} \(\implies\) \(\vert\)A\(\vert\)
        \item A = \{1, 2, 3, 4\} \(\land\) B = \{5, 6, 7\} \(\implies\) \(\vert\)A\(\vert\)\(\times\)\(\vert\)B\(\vert\) = 12
    \end{enumerate}\end{enumerate}
    \item\textbf{Function: }A Function, typically named with a lowercase letter, is a mapping/rule from a set called the domain D to a set called the codomain C that maps each element x in the domain to one element f(x) in the codomain. f: D\(\mapsto\)C.
    \begin{enumerate}\begin{enumerate}
        \item \((f(x) = x^2)\) \(\implies\) (D = $\mathbb{R}$ \(\land\) C = $\mathbb{R}$)
    \end{enumerate}\end{enumerate}
    \item\textbf{Range: }A functions range R is a set \(\vert\) \(\forall\)x \(\in\) D \(\exists\)y \(\in\) R \(\vert\) \(f(\)x\() =\) y.
    \begin{enumerate}\begin{enumerate}
            \item \(f(\)x\()\) = x\(^2\) \(\land\) D(\(f\)) = \{1, 2, 3\} \(\implies\) R(\(f\)) = \{1, 4, 9\)\}
    \end{enumerate}\end{enumerate}
    \item\textbf{Injective: }A function \(f\): D\(\mapsto\)C is injective \(\iff\) \(\forall\)\((a, b)\) \(\in\) D, \(a\) \(\neq\) \(b\) \(\implies\) \(f(a)\) \(\neq\) \(f(b)\).
    \begin{enumerate} \begin{enumerate}
            \item Injective: \(f(\)x\(\))\) = x
            \item Injective: \(f(\)x) = x\(^3\)
            \item Not Injective: \(f(\)x) = y
            \item Not Injective: \(f(\)x) = a person's name where D$(f)$ = the set of all people. This is true because while people have many different names, some of them share the same name.
    \end{enumerate}\end{enumerate}
    \item\textbf{Surjective: }A function \(f\): D\(\mapsto\)C is surjective \(\iff\) \(\forall\)y \(\in\) C \(\exists\)x \(\in\) D \(\vert\) \(f(\)x\()\) = y. (The range of \(f\) = the codomain of \(f\)).
    \begin{enumerate} \begin{enumerate}
            \item Surjective: \(f\): D\(\mapsto\)C where D = the set of students at Carthage and C = the set of student IDs at Carthage. \(f(\)x\()\) = the student ID of a given student x.
            \item Surjective: \(f(\)x\(\))\) = x. D\((f)\) = C\((f)\) = R\((f)\)
            \item Not Surjective: \(f\)(x) = x\(^2\)
            \item Not Surjective: \(f\): D\(\mapsto\)C where D = the set of students at Carthage and C = $\mathbb{N}$. \(f(\)x\()\) = the student ID of a given student x. The set of valid student IDs at carthage is a subset of the natural numbers.
    \end{enumerate}\end{enumerate}
\end{enumerate}
\subsection{Assignments}
\begin{enumerate}
    \item Quiz on monday there will be 2 questions. Question 1: Define and give an example of the given terms. Question 2:  given the sets A, B, C and the universal set U, find the following.
    \begin{enumerate}
        \item What is a set? (capital letters, \{\})
        \item What is an element? (\(\in\))
        \item What is the empty set or null set? (\(\emptyset\))
        \item What is the union of two sets? (\(\cup\))
        \item What is the intersection of two sets? (\(\cap\))
        \item What is the not operation? (-)
        \item What's the complement of a set? (')
        \item Natural numbers? ($\mathbb{N}$)
        \item Integers? ($\mathbb{Z}$)
        \item Rational Numbers? ($\mathbb{Q}$)
        \item Irrational numbers? ($\mathbb{I}$)
        \item Real Numbers? ($\mathbb{R}$)
        \item Complex Numbers? ($\mathbb{C}$)
    \end{enumerate}
    \item Study for Quiz, -definitions/examples. -problems
    \item Do the Cartesian product of AxB for A = \{1, 2, 3, 4\}, B = \{1, 2, 3\}, and compute \(\vert\) A\(\vert\), \(\vert\)B\(\vert\), and \(\vert\)AxB\(\vert\).
    \item come up with 2 functions, 1 involving numbers and one not. ask yourself are they injective / surjective?
\end{enumerate}
\section{Day 3, 1-8-2024}
\subsection{General}
\begin{enumerate}
    \item \(f\) $\mathbb{R}$\(\mapsto\)$\mathbb{R}$
\end{enumerate}
\subsection{Definitions}
\begin{enumerate}
    \item\textbf{Bijective: }a function is Bijective if it is Surjective and Injective.

    \item f:D\(\mapsto\)C is 1:1 if and only if whenever a, b \(\in\) D and a \(\neq\) b then \(f\)(a) \neq \(f\)(b).

    \item \textbf{Horizontal Line Test: }If a graph y = \(f\)(x) of a function \(f\)(x) can only be intersected by one or no points using any horizontal line then the function \(f\)(x) is 1:1.

    \item if \(f\)(x) is a 1:1 function, its inverse function \(f^-1\)(x) is the function such that if I (\(f * f^-1\)) = (\(f^-1\)(x)) = x and (\(f^-1 * f\))(x) = \(f^-1\)(\(f\)(x)) = x. i.e. the inverse \(f^-1\)(x) of a function \(f\)(x) should be another function that does the ``opposite'' or ``undoes'' the origial function.
    \begin{enumerate}
        \item \(g\)(x) = x\(^2\). g: R\(\mapsto\)R
    \end{enumerate}

    \item \(f^-1\)(x) = \(\frac{x-3}{2}\) 1. write f(x) = y. 2. switch x with y. 3. solve for y in terms of x, i.e. get y on one side of the equation by itself and on the other side of equation have everything written in terms of x.

    \item a function f: D\(\mapsto\)C is onto if and only if f(D) = range.

    \item f: R\(\mapsto\)R = C \(\neq\) range R. f(x) = x\(^2\). D = $\mathbb{R}$, f(D) = f($$

    \item trig functions are functions applied to angles.

    \item Trigonometric Functions: Defined two ways:
    \begin{enumerate}
        \item using right triangles:
        \begin{enumerate}
            \item right triangle = triangle with a 90 degree angle
            \item \(\theta\), opp
        \end{enumerate}
        \item using unit circles: -def x2 + y2 = 1 = unit circle = circle of radius 1 centered at the origin
        \item Sine: y = sine of theta, x = cos of theta, as an angle of the unit circle. 30 = (sqrt(3,2), 1/2) 60 = 1/2, sqrt(5/2)
        \item Sine: sin(\(\theta\)) \(\rightarrow\) y
        \item Cosine: cos(\(\theta\)) \(\rightarrow\) x
        \item tangent: tan(\(\theta\)) = \(\frac{sin(\theta)}{cos(\theta)}\)
        \item cotangent: cot(\(\theta\)) = \(\frac{cos(\theta)}{sin(\theta)}\)
        \item cosecant: csc(\(\theta\)) = \(\frac{1}{sin(\theta)}\)
        \item secant: sec(\(\theta\)) = \(\frac{1}{cos}\)
    \end{enumerate}
\end{enumerate}
\subsection{Homework}
\begin{enumerate}
    \item f(x) [2, infinity)\(\mapsto\)R f(x) = sqrt(x - 2)
    \item g: R->R, g(x) = 2x\(^6\)
    \item h: R->R, h(x) = 5x + 4.
    \item 
\end{enumerate}
\section{Day 4, 1-9-2024}
\subsection{General: }
\begin{enumerate}
    \item Plot[Sin[x], \{x, 0, 2 Pi\}]
    \item Solve[]
    \item Degrees to Radians: \(\frac{d\pi}{180}\) = radians
    \item Radians to Degrees: \(\frac{r180}{\pi}\) = degrees
\end{enumerate}
\subsection{Definitions: }
\begin{enumerate}
    \item 
\end{enumerate}
\subsection{Homework: }
\item Monday, January \(15^th\) is Martin Luther King Day. We will not have class so we can participate in the planned MLK events on campus. Instead on that day, be ready to give a short report on any non-Western or female mathematician of your choice on Tuesday, January \(15^th\). -location -time period -education -what are they known for mathematically.
\begin{enumerate}
    \item Quiz Tomorrow:
    \begin{enumerate}
        \item Define and give an example of the given term
        \item Evaluate an angle for a particular trig function.
        \item Determine if a given function is 1:1, onto, both, or neither.
        \begin{enumerate}
            \item function: a function f takes a value x in the domain D and maps it to another value f(x) in C.
            \item domain
            \item codomain
            \item range
            \item sin(\(\theta\)) = y = \(\frac{opp}{hyp}\)
            \item cos(\(\theta\)) = x = \(\frac{adj}{hyp}\)
            \item tan(\(\theta\)) = \(\frac{sin}{cos}\) = \(\frac{opp}{adj}\)
            \item cot(\(\theta\)) = \(\frac{cos}{sin}\) = \(\frac{adj}{opp}\)
            \item sec(\(\theta\)) = \(\frac{1}{cos}\) = \(\frac{hyp}{adj}\)
            \item csc(\(\theta\)) = \(\frac{1}{sin}\) = \(\frac{hyp}{opp}\)
        \end{enumerate}
    \end{enumerate}
    \item look at graphs of what trig functions look like
\end{enumerate}
\section{Day 5, 1-1-2024}
\subsection{General: }
\begin{enumerate}
    \item asmitope? asymptotes
    \item 
\end{enumerate}
\subsection{Definitions: }
\begin{enumerate}
    \item Trigonometric Functions
    \begin{enumerate}
        \item Definitions through standard right triangles, or unit circle.
        \item Graphs
        \item Evaluate the trig functions for standard angles.
        \item identities(relationship between the trig functions and/or right triangles)
    \end{enumerate}
    \item Identities from Trig. to Know:
    \begin{enumerate}
        \item Unit Circle Definitions of Sine and Cosine.
        \item Right Triangle Definitions of the trig functions.
        \item Pythagorean Identities
        \begin{enumerate}
            \item \(\sin^2\theta + \cos^2\theta = 1\) = \((\sin\theta^2) + \cos(\theta^2)\)
            \item \(\tan^4 3\theta = (\tan(3\theta))^4\)
            \item \(\frac{(\sin\theta)^2}{(\sin\theta)^2} + \frac{(\cos\theta)^2}{(\sin\theta)^2} = \frac{1}{(\sin\theta)^2}, 1 + (\cot\theta)^2 = (\csc\theta^2)\)
            \item \(\frac{(\sin\theta)^2}{(\cos\theta)^2} + \frac{(\cos\theta)^2}{(\cos\theta)^2} = \frac{1}{(\cos\theta)^2} = \tan^2\theta + 1 = \sec^2\theta\)
            \item 
        \end{enumerate}
        \item Double Angle Identity
        \begin{enumerate}
            \item \(\sin^2\theta = \frac{1 - \cos(2\theta)}{2}\)
            \item \(\cos^2\theta = \frac{1 + \cos(2\theta)}{2} = (\cos\theta)^2\)
            \item 
        \end{enumerate}
        \item Sum/Difference Formulas
        \begin{enumerate}
            \item \(\sin(a \pm b) = \sin(a)\cos(b) \pm \cos(a)\sin(b)\)
            \item \(\cos(a \pm b) = \cos(a)\cos(b) \mp \sin(a)\sin(b)\)
            \item Ex.1 \(\sin(75\deg) = \sin(30\deg + 45\deg) = \sin(30\deg)\cos(45\deg)+\sin(45\deg)\cos(30\deg)\)
        \end{enumerate}
    \end{enumerate}
\end{enumerate}
\subsection{Assignments: }
\begin{enumerate}
    \item Quiz Tomorrow: Draw all trig functions!
    \item Work on presentation for Tuesday, Jan. 16th. No class on Monday.
\end{enumerate}

\section{Day 7, 1-17-2024}
\subsection{Assignments: }
\begin{enumerate}
    \item Quiz: computation with logs and exponents, graph exponential function and corresponding log.
\end{enumerate}
\section{Day 8, 1-18-2024}
\subsection{Assignments: }
\begin{itemize}
    \item \(\log _4 (80) = 2\frac{a}{a} + a\)
\end{itemize}
\end{document} 