\documentclass[12pt]{article}
\usepackage[margin=1in]{geometry} 
\usepackage{amsmath,amsthm,amssymb,graphicx,mathtools,tikz,hyperref, enumitem}
\usetikzlibrary{positioning}
\newcommand{\n}{\mathbb{N}}
\newcommand{\z}{\mathbb{Z}}
\newcommand{\q}{\mathbb{Q}}
\newcommand{\cx}{\mathbb{C}}
\newcommand{\real}{\mathbb{R}}
\newcommand{\field}{\mathbb{F}}
\newcommand{\ita}[1]{\textit{#1}}
\newcommand{\com}[2]{#1\backslash#2}
\newcommand{\oneton}{\{1,2,3,...,n\}}
\newcommand\idea[1]{\begin{gather*}#1\end{gather*}}
\newcommand\ef{\ita{f} }
\newcommand\eff{\ita{f}}
\newcommand\proofs[1]{\begin{proof}#1\end{proof}}
\newcommand\inv[1]{#1^{-1}}
\newcommand\setb[1]{\{#1\}}
\newcommand\en{\ita{n }}
\newcommand{\vbrack}[1]{\langle #1\rangle}
\newenvironment{theorem}[2][Theorem]{\begin{trivlist}
\item[\hskip \labelsep {\bfseries #1}\hskip \labelsep {\bfseries #2.}]}{\end{trivlist}}
\newenvironment{lemma}[2][Lemma]{\begin{trivlist}
\item[\hskip \labelsep {\bfseries #1}\hskip \labelsep {\bfseries #2.}]}{\end{trivlist}}
\newenvironment{exercise}[2][Exercise]{\begin{trivlist}
\item[\hskip \labelsep {\bfseries #1}\hskip \labelsep {\bfseries #2.}]}{\end{trivlist}}
\newenvironment{reflection}[2][Reflection]{\begin{trivlist}
\item[\hskip \labelsep {\bfseries #1}\hskip \labelsep {\bfseries #2.}]}{\end{trivlist}}
\newenvironment{proposition}[2][Proposition]{\begin{trivlist}
\item[\hskip \labelsep {\bfseries #1}\hskip \labelsep {\bfseries #2.}]}{\end{trivlist}}
\newenvironment{corollary}[2][Corollary]{\begin{trivlist}
\item[\hskip \labelsep {\bfseries #1}\hskip \labelsep {\bfseries #2.}]}{\end{trivlist}}
\hypersetup{colorlinks, linkcolor=blue}

\begin{document}
\date{January \(31^{st}\) 2024}
\title{PHL 1000 notes.}
\author{Ryan Heffelman\\Paul Ulrich \\
PHL 1000 \\Introduction to Philosophy} 
\maketitle

\section{Day 1, Wednesday, 1-31-2024}

\subsection{What does Kant mean by ``nonage'' What is a self-imposed nonage?}
The suffix ``age'' means ``to do something'' and then ``non'' is negation, so ``nonage'' is the inability to do something. If ``nonage'' is the state of being unable to do things, a ``self-imposed nonage'' is simply the self-imposition of the state of nonage.

\subsection{Why do people try to get out of their nonage? What are different kinds of factors that make it difficult to get out of?}
Their own drive-- they have nothing to do living in a state of nonage-- no need to exert themselves. A man does not want to live a life of incompetence. 

One of the factors that make escaping nonage difficult is the cattle analogy kant uses, that guardians have made the guarded incompetent, and when they try to do things on their own, they fail, discouraging any further attempts. Another factor, is that he is used to and accustomed to living in his nonage.



\subsection{Kant says that ``the public use of one's reason must be free at all times.'' What does he mean by ``public use''? And how does he explain his assertion that ``the private use of reason may frequently be narrowly restricted''?}

I think he means that people must often censor what they claim to think, or speak only a small acceptable fraction of what they think.

\subsection{Ultimately, do you think that the free public use of reason and the restricted private use can coexist easily enough? Why or why not? Try to think this through as far as you can.}

I think it fundamentally causes conflict

\end{document} 