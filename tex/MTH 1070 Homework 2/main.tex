\documentclass[16pt]{article}
\usepackage[margin=0.5in]{geometry} 
\usepackage{amsmath,amsthm,amssymb,graphicx,mathtools,tikz,hyperref, enumitem}
\usetikzlibrary{positioning}
\newcommand{\n}{\mathbb{N}}
\newcommand{\z}{\mathbb{Z}}
\newcommand{\q}{\mathbb{Q}}
\newcommand{\cx}{\mathbb{C}}
\newcommand{\real}{\mathbb{R}}
\newcommand{\field}{\mathbb{F}}
\newcommand{\ita}[1]{\textit{#1}}
\newcommand{\com}[2]{#1\backslash#2}
\newcommand{\oneton}{\{1,2,3,...,n\}}
\newcommand\idea[1]{\begin{gather*}#1\end{gather*}}
\newcommand\ef{\ita{f} }
\newcommand\eff{\ita{f}}
\newcommand\proofs[1]{\begin{proof}#1\end{proof}}
\newcommand\inv[1]{#1^{-1}}
\newcommand\setb[1]{\{#1\}}
\newcommand\en{\ita{n }}
\newcommand{\vbrack}[1]{\langle #1\rangle}
\newenvironment{theorem}[2][Theorem]{\begin{trivlist}
\item[\hskip \labelsep {\bfseries #1}\hskip \labelsep {\bfseries #2.}]}{\end{trivlist}}
\newenvironment{lemma}[2][Lemma]{\begin{trivlist}
\item[\hskip \labelsep {\bfseries #1}\hskip \labelsep {\bfseries #2.}]}{\end{trivlist}}
\newenvironment{exercise}[2][Exercise]{\begin{trivlist}
\item[\hskip \labelsep {\bfseries #1}\hskip \labelsep {\bfseries #2.}]}{\end{trivlist}}
\newenvironment{reflection}[2][Reflection]{\begin{trivlist}
\item[\hskip \labelsep {\bfseries #1}\hskip \labelsep {\bfseries #2.}]}{\end{trivlist}}
\newenvironment{proposition}[2][Proposition]{\begin{trivlist}
\item[\hskip \labelsep {\bfseries #1}\hskip \labelsep {\bfseries #2.}]}{\end{trivlist}}
\newenvironment{corollary}[2][Corollary]{\begin{trivlist}
\item[\hskip \labelsep {\bfseries #1}\hskip \labelsep {\bfseries #2.}]}{\end{trivlist}}
\hypersetup{colorlinks, linkcolor=blue}

\begin{document}
\Large
\date{}
\title{MTH 1070 Homework 2}
\author{Ryan Heffelman\\Aaron Trautwein \\
MTH 1070 \\Functions, Analysis, and Graphs. \\ January \(5^{th}\) 2024} 
\maketitle

\section{Cartesian Products:}
\begin{enumerate}
    \item If a set A = \{1, 2, 3, 4\}
    \item And a set B = \{1, 2, 3\}
    \item Then the set A\(\times\)B = \{(1, 1), (1, 2), (1, 3), (2, 1), (2, 2), (2, 3), (3, 1), (3, 2), (3, 3), (4, 1), (4, 2), (4, 3)\}
    \item Additionally \(\vert \)A\(\vert = 4\), \(\vert\)B\(\vert = 3\), and \(\vert\)A\(\times\)B\(\vert = 12\).
\end{enumerate}
\section{My Functions:}
\begin{enumerate}
    \item $f(x) = \frac{x(x+1)}{2}$. $f(x)$ returns the sum of natural numbers 0 to $x$ when D = $\mathbb{N}$ and C = $\mathbb{N}$.
    \begin{itemize}
        \item This function is injective.
        \item This function is not surjective, for instance x = 2 \(\implies\) x \(\in\) C \(\land\) x \(\notin\) R.
    \end{itemize}
    \item If a students name is Ryan they receive an A+, if a students name is not Ryan they receive an F. D is the set of students in this classroom, and C is the set of letter grades F through A+.
    \begin{itemize}
        \item This function is not injective as most elements in D are mapped to the same element in C, that is to say that my function results in most students receiving an F.
        \item This function is not surjective as R = \{F, A+\} whereas C = \{F, D-, D, ... A+\}, therefore R \(\neq\) C.
    \end{itemize}
\end{enumerate}

\end{document}