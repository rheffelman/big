\documentclass[16pt]{article}
\usepackage[margin=0.7in]{geometry} 
\usepackage{amsmath,amsthm,amssymb,graphicx,mathtools,tikz,hyperref, enumitem, tikz, venndiagram}
\NeedsTeXFormat{LaTeX2e}
\ProvidesPackage{venndiagram}[2018/06/07 v1.2 (NLCT) Venn diagrams]
\RequirePackage{xkeyval}
\RequirePackage{tikz}
\RequirePackage{etoolbox}
\usetikzlibrary{positioning}
\newcommand{\n}{\mathbb{N}}
\newcommand{\z}{\mathbb{Z}}
\newcommand{\q}{\mathbb{Q}}
\newcommand{\cx}{\mathbb{C}}
\newcommand{\real}{\mathbb{R}}
\newcommand{\field}{\mathbb{F}}
\newcommand{\ita}[1]{\textit{#1}}
\newcommand{\com}[2]{#1\backslash#2}
\newcommand{\oneton}{\{1,2,3,...,n\}}
\newcommand\idea[1]{\begin{gather*}#1\end{gather*}}
\newcommand\ef{\ita{f} }
\newcommand\eff{\ita{f}}
\newcommand\proofs[1]{\begin{proof}#1\end{proof}}
\newcommand\inv[1]{#1^{-1}}
\newcommand\setb[1]{\{#1\}}
\newcommand\en{\ita{n }}
\newcommand{\vbrack}[1]{\langle #1\rangle}
\newenvironment{theorem}[2][Theorem]{\begin{trivlist}
\item[\hskip \labelsep {\bfseries #1}\hskip \labelsep {\bfseries #2.}]}{\end{trivlist}}
\newenvironment{lemma}[2][Lemma]{\begin{trivlist}
\item[\hskip \labelsep {\bfseries #1}\hskip \labelsep {\bfseries #2.}]}{\end{trivlist}}
\newenvironment{exercise}[2][Exercise]{\begin{trivlist}
\item[\hskip \labelsep {\bfseries #1}\hskip \labelsep {\bfseries #2.}]}{\end{trivlist}}
\newenvironment{reflection}[2][Reflection]{\begin{trivlist}
\item[\hskip \labelsep {\bfseries #1}\hskip \labelsep {\bfseries #2.}]}{\end{trivlist}}
\newenvironment{proposition}[2][Proposition]{\begin{trivlist}
\item[\hskip \labelsep {\bfseries #1}\hskip \labelsep {\bfseries #2.}]}{\end{trivlist}}
\newenvironment{corollary}[2][Corollary]{\begin{trivlist}
\item[\hskip \labelsep {\bfseries #1}\hskip \labelsep {\bfseries #2.}]}{\end{trivlist}}
\hypersetup{colorlinks, linkcolor=blue}

\newcounter{itemnumber}

\begin{document}
\large
\date{}
\title{\Large Assignment 3 Part 2 \\ Ryan Heffelman \\ February 16^\(th\), 2024 \\ MTH 1240 \\ Sara Jensen}
\maketitle
\begin{enumerate}
    \item[\textbf{1.6}]
    \item[]
    \begin{enumerate}
        \item[\#1]
        \item[]
        \begin{enumerate}
            \item[(d)] $A \cup \overline{A} = U$
            \item[(e)] $A - \overline{A} = A$
            \item[(f)] $A - \overline{B} = \{4, 6\}$
            \item[(g)] $\overline{A} - \overline{B} = \{5, 8\}$
            \item[(h)] $\overline{A} \cap B = \{5, 8\}$
            \item[(i)] $\overline{\overline{A} \cap B} = \{0, 1, 2, 3, 4, 6, 7, 9, 10\}$ 
        \end{enumerate}
    \end{enumerate}
    \item[\textbf{1.7}]
    \item[]
    \begin{enumerate}
        \item[\textbf{\#4}] $(A \cup B) - C$ \\
        \begin{venndiagram3sets}
        \fillANotC
        \fillBNotC
        \end{venndiagram3sets}

        \item[\textbf{\#6}] $A \cap (B \cup C)$ \\
        \begin{venndiagram3sets}
        \fillACapC
        \fillACapB
        \end{venndiagram3sets}
        \item[] $(A \cap B) \cup (A \cap C)$ \\
        \begin{venndiagram3sets}
        \fillACapC
        \fillACapB
        \end{venndiagram3sets}
        \item[] Yes, they look to be equivalent. \\
        \item[\textbf{\#13}] $(A \cup B \cup C) - (A \cap B \cap C)$
        \item[\textbf{\#14}] $(A - (B \cup C)) \cup (A \cap B \cap C)$ 
    \end{enumerate}
    
\end{enumerate}

\end{document}