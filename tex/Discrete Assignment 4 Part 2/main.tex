\documentclass[16pt]{article}
\usepackage[margin=0.7in]{geometry} 
\usepackage{amsmath, amsthm, amssymb, mathrsfs, graphicx, mathtools, tikz, hyperref, enumitem}
\usetikzlibrary{positioning}
\newcommand{\n}{\mathbb{N}}
\newcommand{\z}{\mathbb{Z}}
\newcommand{\q}{\mathbb{Q}}
\newcommand{\cx}{\mathbb{C}}
\newcommand{\real}{\mathbb{R}}
\newcommand{\field}{\mathbb{F}}
\newcommand{\ita}[1]{\textit{#1}}
\newcommand{\com}[2]{#1\backslash#2}
\newcommand{\oneton}{\{1,2,3,...,n\}}
\newcommand\idea[1]{\begin{gather*}#1\end{gather*}}
\newcommand\ef{\ita{f} }
\newcommand\eff{\ita{f}}
\newcommand\proofs[1]{\begin{proof}#1\end{proof}}
\newcommand\inv[1]{#1^{-1}}
\newcommand\setb[1]{\{#1\}}
\newcommand\en{\ita{n }}
\newcommand{\vbrack}[1]{\langle #1\rangle}
\newenvironment{theorem}[2][Theorem]{\begin{trivlist}
\item[\hskip \labelsep {\bfseries #1}\hskip \labelsep {\bfseries #2.}]}{\end{trivlist}}
\newenvironment{lemma}[2][Lemma]{\begin{trivlist}
\item[\hskip \labelsep {\bfseries #1}\hskip \labelsep {\bfseries #2.}]}{\end{trivlist}}
\newenvironment{exercise}[2][Exercise]{\begin{trivlist}
\item[\hskip \labelsep {\bfseries #1}\hskip \labelsep {\bfseries #2.}]}{\end{trivlist}}
\newenvironment{reflection}[2][Reflection]{\begin{trivlist}
\item[\hskip \labelsep {\bfseries #1}\hskip \labelsep {\bfseries #2.}]}{\end{trivlist}}
\newenvironment{proposition}[2][Proposition]{\begin{trivlist}
\item[\hskip \labelsep {\bfseries #1}\hskip \labelsep {\bfseries #2.}]}{\end{trivlist}}
\newenvironment{corollary}[2][Corollary]{\begin{trivlist}
\item[\hskip \labelsep {\bfseries #1}\hskip \labelsep {\bfseries #2.}]}{\end{trivlist}}
\hypersetup{colorlinks, linkcolor=blue}

\newcounter{itemnumber}

\begin{document}
\large
\date{}
\title{\Large Assignment 4 Part 2 \\ Ryan Heffelman \\ February 28^\(th\), 2024 \\ MTH 1240 \\ Sara Jensen}
\maketitle
\begin{enumerate}
    \item[\textbf{1.}] Suppose $x$ and $y$ are integers. If $x$ and $y$ are odd then $xy$ is odd.
    \item[\textbf{def-}] An integer $n$ is odd if $n = 2a + 1$ for some integer a.
    \item[\textbf{Proof:}] Assume $x$ and $y$ are odd. So $x = 2a + 1$ and $y = 2b + 1$ as per the definition of an odd number. So $x \cdot y = (2a + 1) \cdot (2b + 1)$. Expanding the expression we have \\$x \cdot y = 4ab + 2a + 2b + 1$ \\= $2(2ab + a + b) + 1$ \\  If we rewrite this expression substituting $2ab + a + b$ with a new variable $c$ where $c \in \mathbb{Z}$ we get $2c + 1$. Note that $(2ab + a + b) \in \mathbb{Z}$ because the integers are closed under multiplication and addition, so it is possible to rewrite it this way. Therefore, $xy$ where $x$ and $y$ are odd is exactly equivalent to the definition of an odd number, because it can be rewritten in the form $xy = 2c + 1$. \square
\end{enumerate}
\\ \\
\begin{enumerate}
    \item[\textbf{2.}] Suppose that $A \subseteq B$. Prove that $A \oplus B = B - A$
    \item[\textbf{def-}] The set $A$ is a subset of a set $B$ if all elements of $A$ are also elements of $B$.
    \item[\textbf{Proof:}] Let $c$ be an abstract element in $U$ where $(A \cup B) \in U$.\\
    We get the following truth table regarding $c$'s membership to $A$, $B$, $A \oplus B$, and $B - A$.
    
    \begin{tabular}{|c|c|c|c|}
    \hline
    $c \in A$ & $c \in B$ & $c \in (A \oplus B)$ & $c \in (B - A)$ \\
    \hline
    $T$ & $T$ & $F$ & $F$ \\
    \hline
    $T$ & $F$ & $T$ & $F$ \\
    \hline
    $F$ & $T$ & $T$ & $T$ \\
    \hline
    $F$ & $F$ & $F$ & $F$ \\
    \hline
    \end{tabular}
    \item[] In this table there is one row where the truth values of $c \in (A \oplus B)$ and $c \in (B - A)$ are not equivalent. In this row $c \in A \equiv T$ and $c \in B \equiv F$. This row is impossible per the definition of a subset, $c$ cannot be an element of $A$ and not be an element of $B$ because all elements in $A$ must be in $B$. \\Here is a table with all \textbf{possible} scenarios:\\
    \begin{tabular}{|c|c|c|c|}
    \hline
    $c \in A$ & $c \in B$ & $c \in (A \oplus B)$ & $c \in (B - A)$ \\
    \hline
    $T$ & $T$ & $F$ & $F$ \\
    \hline
    $F$ & $T$ & $T$ & $T$ \\
    \hline
    $F$ & $F$ & $F$ & $F$ \\
    \hline
    \end{tabular}
    \item[] As you can see, the truth values of $c \in (A \oplus B)$ and $c \in (B - A)$ are all equivalent in each row for the abstract element $c$. Therefore $A \oplus B = B - A$ where $A \subseteq B$. \square
\end{enumerate}

\end{document}